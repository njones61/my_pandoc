% Options for packages loaded elsewhere
\PassOptionsToPackage{unicode}{hyperref}
\PassOptionsToPackage{hyphens}{url}
\documentclass[
]{article}
\usepackage{xcolor}
\usepackage{amsmath,amssymb}
\setcounter{secnumdepth}{-\maxdimen} % remove section numbering
\usepackage{iftex}
\ifPDFTeX
  \usepackage[T1]{fontenc}
  \usepackage[utf8]{inputenc}
  \usepackage{textcomp} % provide euro and other symbols
\else % if luatex or xetex
  \usepackage{unicode-math} % this also loads fontspec
  \defaultfontfeatures{Scale=MatchLowercase}
  \defaultfontfeatures[\rmfamily]{Ligatures=TeX,Scale=1}
\fi
\usepackage{lmodern}
\ifPDFTeX\else
  % xetex/luatex font selection
\fi
% Use upquote if available, for straight quotes in verbatim environments
\IfFileExists{upquote.sty}{\usepackage{upquote}}{}
\IfFileExists{microtype.sty}{% use microtype if available
  \usepackage[]{microtype}
  \UseMicrotypeSet[protrusion]{basicmath} % disable protrusion for tt fonts
}{}
\makeatletter
\@ifundefined{KOMAClassName}{% if non-KOMA class
  \IfFileExists{parskip.sty}{%
    \usepackage{parskip}
  }{% else
    \setlength{\parindent}{0pt}
    \setlength{\parskip}{6pt plus 2pt minus 1pt}}
}{% if KOMA class
  \KOMAoptions{parskip=half}}
\makeatother
\usepackage{longtable,booktabs,array}
\usepackage{calc} % for calculating minipage widths
% Correct order of tables after \paragraph or \subparagraph
\usepackage{etoolbox}
\makeatletter
\patchcmd\longtable{\par}{\if@noskipsec\mbox{}\fi\par}{}{}
\makeatother
% Allow footnotes in longtable head/foot
\IfFileExists{footnotehyper.sty}{\usepackage{footnotehyper}}{\usepackage{footnote}}
\makesavenoteenv{longtable}
\usepackage{graphicx}
\makeatletter
\newsavebox\pandoc@box
\newcommand*\pandocbounded[1]{% scales image to fit in text height/width
  \sbox\pandoc@box{#1}%
  \Gscale@div\@tempa{\textheight}{\dimexpr\ht\pandoc@box+\dp\pandoc@box\relax}%
  \Gscale@div\@tempb{\linewidth}{\wd\pandoc@box}%
  \ifdim\@tempb\p@<\@tempa\p@\let\@tempa\@tempb\fi% select the smaller of both
  \ifdim\@tempa\p@<\p@\scalebox{\@tempa}{\usebox\pandoc@box}%
  \else\usebox{\pandoc@box}%
  \fi%
}
% Set default figure placement to htbp
\def\fps@figure{htbp}
\makeatother
\ifLuaTeX
  \usepackage{luacolor}
  \usepackage[soul]{lua-ul}
\else
  \usepackage{soul}
\fi
\setlength{\emergencystretch}{3em} % prevent overfull lines
\providecommand{\tightlist}{%
  \setlength{\itemsep}{0pt}\setlength{\parskip}{0pt}}
\usepackage{bookmark}
\IfFileExists{xurl.sty}{\usepackage{xurl}}{} % add URL line breaks if available
\urlstyle{same}
\hypersetup{
  hidelinks,
  pdfcreator={LaTeX via pandoc}}

\author{}
\date{}

\begin{document}

Article

GWBASE -- An Algorithm for Assessing the Impact of Groundwater Decline
on Baseflow in US Streams

Xueyi Li\textsuperscript{1}, Norman L. Jones\textsuperscript{1,}* and
Gustavious P. Williams\textsuperscript{1}, Amin
Aghababaei\textsuperscript{1}, Riley C. Hales\textsuperscript{1}

\begin{longtable}[]{@{}
  >{\raggedright\arraybackslash}p{(\linewidth - 0\tabcolsep) * \real{1.0000}}@{}}
\toprule\noalign{}
\begin{minipage}[b]{\linewidth}\raggedright
Academic Editor: Firstname Lastname

Received: date

Revised: date

Accepted: date

Published: date

\textbf{Citation:} To be added by editorial staff during production.

\textbf{Copyright:} © 2025 by the authors. Submitted for possible open
access publication under the terms and conditions of the Creative
Commons Attribution (CC BY) license
(https://creativecommons.org/licenses/by/4.0/).
\end{minipage} \\
\midrule\noalign{}
\endhead
\bottomrule\noalign{}
\endlastfoot
\end{longtable}

\textsuperscript{1} Civil and Construction Engineering, Brigham Young
University;
\href{mailto:xueyil@student.byu.edu}{\nolinkurl{xueyil@student.byu.edu}}
(X.L.); \href{mailto:njones@byu.edu}{\nolinkurl{njones@byu.edu}}
(N.L.J.);
\href{mailto:gus.p.williams@byu.edu}{\nolinkurl{gus.p.williams@byu.edu}}
(G.P.W.)\textsuperscript{;}
\href{mailto:aghababa@student.byu.edu}{\nolinkurl{aghababa@student.byu.edu}}
(A.A.); \href{mailto:rchales@byu.edu}{\nolinkurl{rchales@byu.edu}}
(R.C.H)

\textbf{*} Correspondence: njones@byu.edu; Tel.: (+1-801-422-7569)

\textbf{Abstract}

A single paragraph of about 200 words maximum. For research articles,
abstracts should give a pertinent overview of the work. We strongly
encourage authors to use the following style of structured abstracts,
but without headings: (1) Background: Place the question addressed in a
broad context and highlight the purpose of the study; (2) Methods:
briefly describe the main methods or treatments applied; (3) Results:
summarize the article's main findings; (4) Conclusions: indicate the
main conclusions or interpretations. The abstract should be an objective
representation of the article and it must not contain results that are
not presented and substantiated in the main text and should not
exaggerate the main conclusions.

\textbf{Keywords:} groundwater, baseflow, drought

\begin{enumerate}
\def\labelenumi{\arabic{enumi}.}
\item
  Introduction
\end{enumerate}

Baseflow represents the portion of streamflow that is sustained
primarily by groundwater discharge during periods of little or no
precipitation. It plays a critical role in maintaining streamflow
continuity, supporting aquatic ecosystems, and regulating water quality.
Because baseflow reflects the long-term balance between groundwater
recharge and discharge, understanding its variability is fundamental to
evaluating watershed resilience under changing climatic and
anthropogenic conditions.

Groundwater can be a major contributor to baseflow in many basins,
especially in regions with shallow water tables and permeable
hydrogeologic formations. Numerous studies have demonstrated that
reductions in groundwater storage can directly diminish baseflow,
leading to streamflow depletion, reduced ecosystem health, and altered
hydrologic regimes {[}1{]}{[}2{]}{[}3{]}. Recent research has documented
widespread declines in groundwater levels across parts of the United
States, particularly in intensively irrigated agricultural regions
{[}4{]}{[}5{]}{[}6{]}. Investigations such as the \emph{New York Times}
national groundwater assessment and recent studies in the Central Valley
of California have highlighted significant long-term declines in water
tables and associated streamflow reductions {[}7{]}. Other regional
analyses have also reported similar trends in declining groundwater
storage and reduced baseflow contributions, underscoring the urgency of
understanding groundwater--surface water connectivity at large spatial
scales. {[}8{]}

The United States provides an unparalleled opportunity to study these
interactions due to the availability of extensive, high-quality
hydrologic observations. The USGS maintains national networks of both
stream gages and groundwater wells, offering decades of concurrent daily
streamflow and groundwater level data. This data richness makes it
possible to explore groundwater--baseflow relationships systematically
across diverse hydroclimatic and geologic settings.

The objective of this study is to develop a generalized algorithm that
leverages these national datasets to quantify the relationship between
groundwater level variations and baseflow dynamics. Specifically, the
algorithm is designed to (1) identify basins where groundwater and
baseflow are hydrologicallyconnected, and (2) evaluate how declining
groundwater levels influence streamflow during baseflow-dominated
periods. We have implemented this algorithm in a Python package called
GWBASE.

\begin{enumerate}
\def\labelenumi{\arabic{enumi}.}
\item
  Study Area
\end{enumerate}

Although this paper focuses primarily on the methodological development
of the GWBASE algorithm, we demonstrate the application of the algorithm
within the Great Salt Lake Basin (GSLB) in Utah (Figure 1). The GSLB is
a closed hydrologic system in which all surface water drains toward the
Great Salt Lake, a terminal lake with no outlet to the ocean. Streamflow
is supplied mainly by snowmelt-fed rivers rising in the Wasatch and
Uinta mountains, while evaporation accounts for most losses. Annual
precipitation ranges from roughly 10--65 cm in the low-elevation valleys
to more than 100 cm in the surrounding mountain headwaters, producing
strong hydroclimatic gradients and sustained groundwater--surface water
exchange {[}9{]}.

\begin{quote}
\includegraphics[width=3.61043in,height=5.33786in]{gwbase_manuscript_07/media/image1.png}
\end{quote}

Figure 1. Great Salt Lake Basin (GSLB) study area

The basin has an effective hydrologic area of about 55,000 km² and
supports nearly two million residents concentrated in the Salt Lake
City--Ogden urban corridor {[}10{]}. The lake and its tributaries play
key ecological and economic roles, and recent declines in lake level
have heightened concern about long-term reductions in inflows. This
setting offers a relevant case study for evaluating how groundwater
level changes may influence baseflow, and how a national-scale framework
could support assessments of groundwater contributions to streamflow
under changing hydrologic conditions.

\begin{enumerate}
\def\labelenumi{\arabic{enumi}.}
\setcounter{enumi}{1}
\item
  Data
\end{enumerate}

This study integrates two national‐scale datasets: (1) daily streamflow
records from the United States Geological Survey (USGS) stream gaging
network, and (2) groundwater level observations from the USGS National
Groundwater Monitoring Network (NGWMN). Stream gages were used to
delineate streamflow regimes and identify periods of baseflow dominance,
while groundwater wells were analyzed to evaluate temporal variations in
water table elevation (WTE).

\begin{enumerate}
\def\labelenumi{\arabic{enumi}.}
\item
  Groundwater Level Data
\end{enumerate}

Groundwater level data were obtained from the United States Geological
Survey (USGS) National Water Information System (NWIS) {[}11{]}. NWIS
provides long-term observations of groundwater levels from over 850,000
monitoring wells across the United States and contains millions of water
level records. Each record includes well location, elevation, and depth
to water table. These data were downloaded and processed to obtain water
table elevations (WTE) referenced to mean sea level.

Within the Great Salt Lake Basin (GSLB), a total of 8752 wells were
identified with usable water-level records. The dataset includes both
continuous and intermittent measurements collected from 1906 - 2023.
Figure 2 shows the spatial distribution of these wells across the basin.
The density of wells varies by subbasin, with higher concentrations in
valley regions and fewer records in upland areas. Together, these wells
provide a spatially extensive representation of groundwater conditions
across the study domain.

\includegraphics[width=3.6in,height=4.16058in]{gwbase_manuscript_07/media/image2.png}

Figure . Wells in GSLB

\begin{enumerate}
\def\labelenumi{\arabic{enumi}.}
\setcounter{enumi}{1}
\item
  Streamflow data and gage information
\end{enumerate}

Daily streamflow data were obtained from the U.S. Geological Survey
National Water Information System {[}12{]}. Observed discharge records
were used as the surface water data source for this study. Stream gages
with insufficient data coverage were excluded from the analysis.

Stream gage location and network information were obtained from the
National Water Model {[}12{]}. Gage locations were linked to stream
reaches in the National Water Model to identify upstream--downstream
relationships and delineate contributing basins. This information was
used to support gage selection and to pair stream gages with nearby
groundwater wells.

\begin{enumerate}
\def\labelenumi{\arabic{enumi}.}
\setcounter{enumi}{2}
\item
  Hydrography Data
\end{enumerate}

Stream network and catchment boundaries used in this study were obtained
from GEOGloWS, which provides globally consistent hydrologic features
{[}13{]}. The GEOGloWS stream network represents the global river system
as a connected set of stream reaches, each with a unique identifier,
geometric attributes, and predefined upstream--downstream relationships.
Corresponding catchment polygons delineate the contributing drainage
area for each stream reach.

Figure 3 presents the spatial layout of major streams and catchments
used in this study. Together with the groundwater well data, these
datasets form the foundation for the well--gage pairing and analysis
described in Section 3.

\begin{quote}
\includegraphics[width=4.39014in,height=4.57153in]{gwbase_manuscript_07/media/image3.png}
\end{quote}

Figure . Streams in GSLB

\begin{enumerate}
\def\labelenumi{\arabic{enumi}.}
\setcounter{enumi}{3}
\item
  \hl{Baseflow classification label data}
\end{enumerate}

Daily streamflow records at each gage were accompanied by a binary
indicator identifying baseflow-dominated conditions. This indicator (BFD
= 1 for baseflow-dominated days and BFD = 0 otherwise) was obtained from
an externally developed machine-learning classification framework
{[}14{]}. In this study, the BFD flag was used directly as an input
dataset to filter streamflow and groundwater observations, isolating
periods when streamflow variability is expected to be primarily
controlled by groundwater discharge rather than surface runoff
processes.

\begin{enumerate}
\def\labelenumi{\arabic{enumi}.}
\setcounter{enumi}{2}
\item
  Methods

  \begin{enumerate}
  \def\labelenumii{\arabic{enumii}.}
  \item
    Overview
  \end{enumerate}
\end{enumerate}

We developed a systematic algorithm to evaluate the relationship between
groundwater level variations and streamflow during baseflow-dominated
(BFD) periods, which we implemented in a Python package called GWBASE.
The GWBASE workflow (Figure 4) integrates spatial pairing between wells
and gages, temporal interpolation of groundwater levels, formation of
water level change vs. baseflow change data pairs (ΔWTE--ΔQ), and a
statistical analysis of ΔWTE--ΔQ relationships. The overall GWBASE
workflow is illustrated in INSERT REF. Each of these steps is described
in detail in the following section.

\begin{quote}
\includegraphics[width=5.236in,height=3.01623in]{gwbase_manuscript_07/media/image4.png}
\end{quote}

Figure . GWBASE workflow

\begin{enumerate}
\def\labelenumi{\arabic{enumi}.}
\setcounter{enumi}{1}
\item
  Algorithm Steps

  \begin{enumerate}
  \def\labelenumii{\arabic{enumii}.}
  \item
    Step 1: Identify Stream Network and Upstream Catchments
  \end{enumerate}
\end{enumerate}

In the first step, stream gages are first matched to catchments by
placing each gage's coordinates inside the subbasin polygons. This
provides the basic gage--catchment links. In many drainage systems,
several gages can exist along the same flow path. If each gage are
analyzed separately, their upstream areas would be processed repeatedly.
To avoid this, the workflow identifies one terminal gage for each
drainage network. A terminal gage is defined here as a gage that has no
other gage located downstream.

To identify these terminal locations, the stream network is converted
into a directed graph using the catchment connectivity fields in the
hydrographic dataset. The upstream catchment ID (LINKNO) and downstream
link (DSLINKNO) form the graph structure. Using this representation,
downstream paths are checked for every gage-associated catchment. Gages
with no downstream path to another gage are classified as terminal. The
initial list of terminals is then reviewed and adjusted using hydrologic
knowledge to correct for classification errors.

After the terminal gages are determined, their upstream contributing
areas are delineated. This is done by tracing all catchments that drain
to each terminal gage through the directed network. The result is a set
of complete and non-overlapping upstream catchment groups. Groundwater
wells are then intersected with these catchments to assign each well to
the terminal gage that receives its drainage. These gage--well
associations serve as the basis for the later groundwater--streamflow
comparison.

Figure 5 illustrates a simple example of how upstream catchments are
identified for a terminal gage. In this network, gage A is the most
downstream location. Along the main flow path, gage E drains catchment 5
and flows into gage D, which drains catchment 4. Both E and D then flow
into gage C, which drains catchment 3. Because C is downstream of both D
and E, its upstream area includes catchments 3, 4, and 5. Gage B,
located in catchment 2, also flows directly to gage A. Consequently, the
total upstream drainage area of gage A consists of catchments 2, 3, 4,
and 5 together with its own local catchment (catchment 1). This example
shows how the upstream catchment set at any gage is formed by combining
its own catchment with those of all upstream gages identified through
the flow network.

\begin{quote}
\includegraphics[width=5.65807in,height=3.05609in]{gwbase_manuscript_07/media/image5.png}
\end{quote}

Figure . Example drainage network showing terminal and upstream gages,
local catchments, and streamflow direction used to delineate
contributing areas.

\begin{enumerate}
\def\labelenumi{\arabic{enumi}.}
\setcounter{enumi}{1}
\item
  Step 2: Locate Groundwater Wells within Catchments
\end{enumerate}

After the terminal drainage areas are defined, groundwater monitoring
wells are spatially matched to the catchments in which they are located.
This is done by intersecting well coordinates with catchment boundaries
and confirming that each well falls within the expected area. The well
locations are then linked to the corresponding terminal gage of the
catchment. This produces a consistent set of well--gage pairs that
reflects the natural watershed structure and provides the basis for
comparing groundwater levels with downstream streamflow.

Figure 6 illustrates the process of locating groundwater wells within a
catchment. All wells situated inside the catchment boundary are
identified through a point-in-polygon operation. This establishes the
set of wells that contribute to the downstream gage and defines the
spatial domain for subsequent groundwater--surface water comparison.

\begin{quote}
\includegraphics[width=5.07214in,height=2.6233in]{gwbase_manuscript_07/media/image6.png}
\end{quote}

Figure .Find catchment area and wells

\begin{enumerate}
\def\labelenumi{\arabic{enumi}.}
\setcounter{enumi}{2}
\item
  Step 3: Associate Wells with Nearest Stream Segments
\end{enumerate}

Each well is further associated with the nearest stream segment to
establish a more detailed hydrologic connection. Using the hydrographic
network, the closest river reach to each well is identified and the
corresponding reach identifier and streambed elevation are recorded.
These attributes allow elevation-based screening in later steps, where
wells with unrealistic vertical separation from the stream are removed.
The resulting well--reach linkage ensures that groundwater levels are
compared with the most physically relevant part of the stream network.

Figure 7 demonstrates how each well is associated with the nearest
stream segment. For every well, the closest reach in the stream network
is identified, and the direction of the nearest-distance link is shown.
This association ensures that each well is connected to a physically
relevant part of the channel network, which is necessary for later
elevation-based screening.

\begin{quote}
\includegraphics[width=4.96941in,height=2.36885in]{gwbase_manuscript_07/media/image7.png}
\end{quote}

Figure 7. Relate wells to stream segments

\begin{enumerate}
\def\labelenumi{\arabic{enumi}.}
\setcounter{enumi}{3}
\item
  Step 4: Filter Wells with Insufficient Data
\end{enumerate}

Groundwater level records are screened to remove well measurements that
lack sufficient temporal coverage. A two-stage outlier check is applied,
first using a Z-score method with a threshold of 3.0 to identify values
that deviate strongly from the mean, and then using an interquartile
range filter with a multiplier of 1.5 to remove observations outside the
expected data spread. These procedures remove extreme measurements that
could affect later interpolation. Wells with fewer than two valid
measurements are excluded because they do not provide enough information
to represent seasonal or interannual variability. Only wells with
adequate data density are retained for further analysis.

\begin{enumerate}
\def\labelenumi{\arabic{enumi}.}
\setcounter{enumi}{4}
\item
  Step 5: Temporal Interpolation of Groundwater Levels
\end{enumerate}

Groundwater level time series are interpolated to daily resolution using
the Piecewise Cubic Hermite Interpolating Polynomial (PCHIP) method.
This approach preserves monotonic patterns between observations and
avoids unrealistic oscillations that can occur with traditional spline
interpolation. As Figure 8 shown below, for each well, observation dates
are converted to numerical time, the PCHIP function is applied between
consecutive measurements, and daily values are generated across the
entire period of record. Local extrema are preserved, and the resulting
time series maintains hydrologic realism. The interpolated results are
then combined with well metadata, including location and surface
elevation, to form a consistent dataset for later comparison with daily
streamflow records.

\begin{quote}
\includegraphics[width=5.63774in,height=1.75338in]{gwbase_manuscript_07/media/image8.png}
\end{quote}

Figure . Example of daily groundwater level interpolation using the
PCHIP method.

Red points represent original groundwater level observations, and the
blue line shows the PCHIP-interpolated daily time series. The method
preserves monotonic trends and local extrema while avoiding artificial
oscillations.

\begin{enumerate}
\def\labelenumi{\arabic{enumi}.}
\setcounter{enumi}{5}
\item
  Step 6: Elevation-Based Filtering
\end{enumerate}

To focus on wells with realistic potential for groundwater and surface
water interaction, an elevation-based screening is applied. The logic is
that well with water levels far below the stream elevation are not
likely to impact baseflow to the stream. For each well, the interpolated
water table elevation is compared with the elevation of the nearest
stream segment identified in previous steps. Wells with water levels far
below the local streambed are removed, since these conditions typically
represent deep or confined aquifers with limited influence on
streamflow. This procedure retained wells where groundwater levels are
close to or higher than the nearby stream channel, reflecting conditions
that can support hydrologic exchange.

Figure 9 shows a simple example of the elevation-based filter using a 30
m buffer. The blue line marks the streambed elevation. Wells plotted in
the blue or green zones fall within 30 m of the streambed and are kept
for analysis. Wells in the red zone lie more than 30 m below the
streambed and are removed. This example illustrates how vertical
separation is used to decide whether a well is likely to interact with
the nearby stream.

\begin{quote}
\includegraphics[width=6.29699in,height=2.69184in]{gwbase_manuscript_07/media/image9.tiff}
\end{quote}

Figure . Conceptual illustration of elevation-based well filtering using
a 30 m buffer.

The blue line represents the streambed elevation. Wells within 30 m of
the streambed (blue and green zones) are retained, while wells more than
30 m below the streambed (red zone) are excluded.

\begin{enumerate}
\def\labelenumi{\arabic{enumi}.}
\setcounter{enumi}{6}
\item
  Step 7: Pair Groundwater and Streamflow Records under
  Baseflow-Dominated Conditions
\end{enumerate}

\hl{In our earlier study, we developed a machine learning classifier to
identify baseflow-dominated days in daily streamflow records}
\hl{{[}14{]}. The classifier labels each day as either
baseflow-dominated or non-baseflow by evaluating streamflow behavior,
and identifying periods when flow is sustained mainly by groundwater
discharge rather than surface runoff.}

For all dates labeled as baseflow-dominated, each well's daily water
table elevation is paired with the streamflow observed on the same day
at the corresponding terminal gage. These paired records represent
hydrologic conditions when streamflow is primarily controlled by
groundwater discharge, making them suitable for assessing
groundwater--surface water connectivity.

\begin{enumerate}
\def\labelenumi{\arabic{enumi}.}
\setcounter{enumi}{7}
\item
  Step 8: Compute ΔWTE and ΔQ
\end{enumerate}

For each well--gage pair, the earliest BFD day was selected as the
baseline condition, with initial water level (\emph{WTE₀}) and discharge
(\emph{Q₀}). Subsequent BFD observations were converted to changes
relative to the baseline as show in Equation 1 and Equation 2 :

\[\begin{array}{r}
_{}\#()
\end{array}\]

\[\begin{array}{r}
_{}\#()
\end{array}\]

To illustrate this procedure, Figure 10 shows an example of a well--gage
pair in which the first day classified as baseflow-dominated (BFD=1) is
identified as the reference condition. On this date, the well's
groundwater elevation (WTE₀) and the gage's stream discharge (Q₀) are
marked with gold star symbols in the upper (WTE) and lower (Q) panels,
respectively. Horizontal dashed lines indicate the baseline values, and
a vertical connector links the two stars, emphasizing that both baseline
quantities correspond to the same hydrologic moment. Periods classified
as BFD=1 are shaded to highlight the subset of observations used in the
subsequent computations. In this zone, each ΔWTE and the corresponding
ΔQ are linked to form a dataset of ΔWTE--ΔQ pairs. These pairs form the
basis of assessing the impact of rising or falling groundwater levels on
baseflow.

\begin{quote}
\includegraphics[width=5.71443in,height=3.83206in]{gwbase_manuscript_07/media/image10.png}
\end{quote}

Figure . Example time series of groundwater level (WTE) and streamflow
(Q) with baseflow-dominated (BFD) periods highlighted.

\begin{enumerate}
\def\labelenumi{\arabic{enumi}.}
\setcounter{enumi}{8}
\item
  Step 9: Analyze ΔWTE--ΔQ Relationships
\end{enumerate}

We use simple linear regression to evaluate the relationship between
ΔWTE and ΔQ for each well--gage pair during baseflow-dominated periods
and to quantify groundwater influence on streamflow. Correlation
measures are used to identify wells that exhibit strong hydrologic
connectivity. Spatial maps are created to visualize areas with high or
low correlation across the study region. In addition, ΔWTE and ΔQ are
aggregated by terminal gage and for the complete drainage network to
assess collective groundwater-driven changes in streamflow.

\begin{enumerate}
\def\labelenumi{\arabic{enumi}.}
\setcounter{enumi}{2}
\item
  Mutual Information Analysis
\end{enumerate}

Mutual information (MI) is used to quantify the statistical dependence
between groundwater level changes (ΔWTE) and streamflow changes (ΔQ)
during baseflow-dominated periods. Unlike linear correlation, MI can
capture both linear and nonlinear relationships and does not assume a
specific functional form between variables.

For two random variables \(\)and \(\), mutual information is defined as:

\[\begin{array}{r}
()\sum_{}^{}{\sum_{}^{}{\left( \frac{()}{()()} \right)}}\#()
\end{array}\]

where \(\)is the joint probability distribution of \(\)and \(\), and
\(\)and \(\) are their marginal distributions. Mutual information
measures the reduction in uncertainty of one variable given knowledge of
the other. A value of zero indicates statistical independence, while
larger values indicate stronger dependence.

In this study, MI is computed between ΔWTE and ΔQ for each well--gage
pair using data from baseflow-dominated days only. MI is used as a
complementary metric to linear regression and correlation, providing
additional insight into groundwater--streamflow connectivity when
relationships may be nonlinear or heterogeneous across space.

\begin{enumerate}
\def\labelenumi{\arabic{enumi}.}
\setcounter{enumi}{3}
\item
  Cross-correlation function (CCF) analysis
\end{enumerate}

The cross-correlation function (CCF) is used to examine the temporal
relationship between groundwater level changes (ΔWTE) and streamflow
changes (ΔQ) during baseflow-dominated periods. CCF quantifies the
similarity between two time series as a function of time lag and is used
to identify delayed groundwater responses in streamflow.

For two time series \(_{}\)and \(_{}\), the cross-correlation at lag
\(\)is defined as:

\[\begin{array}{r}
_{}()\frac{\left(_{}_{} \right)}{_{}_{}}\#()
\end{array}\]

where \(\)is the covariance between the two series, \(_{}\)and
\(_{}\)are their standard deviations, and \(\)represents the time lag.
Positive lag values indicate that changes in groundwater levels precede
changes in streamflow, while negative lags indicate the opposite.

In this study, CCF is computed between ΔWTE and ΔQ for each well--gage
pair using baseflow-dominated days only. The magnitude of the
cross-correlation and the lag at which it peaks are used to assess the
strength and timing of groundwater--streamflow interactions.

\begin{enumerate}
\def\labelenumi{\arabic{enumi}.}
\setcounter{enumi}{4}
\item
  \hl{Machine learning model}
\end{enumerate}

Baseflow-dominated (BFD) periods were identified using a
machine-learning classification framework previously developed {[}14{]}.
The model operates on daily streamflow time series and assigns a binary
label to each day, where BFD = 1 indicates conditions dominated by
groundwater discharge and BFD = 0 denotes periods influenced by surface
runoff or event flow.

The classifier was trained using hydrologically relevant features
derived from streamflow dynamics, enabling it to distinguish
recession-driven baseflow behavior from event-driven responses across a
wide range of hydrologic regimes. Model performance and generalizability
were evaluated in the original study using multiple gages and
independent validation datasets.

In the present work, the resulting BFD classifications were applied to
each gage to filter streamflow and groundwater observations. Subsequent
analyses were restricted to BFD = 1 periods, ensuring that inferred
relationships between groundwater levels and streamflow reflect
baseflow-controlled conditions rather than transient runoff responses.
Figure 11 illustrates an example hydrograph with BFD periods
highlighted.

\includegraphics[width=3.95968in,height=3.03802in]{gwbase_manuscript_07/media/image11.png}

Figure 11. Baseflow Dominant Flows (0-1)

\begin{enumerate}
\def\labelenumi{\arabic{enumi}.}
\setcounter{enumi}{3}
\item
  Results
\end{enumerate}

We use the Great Salt Lake Basin (GSLB) as a case study to demonstrate
our analytical framework. After applying the terminal gage
identification algorithm, we identified 12 terminal gages in GSLB
(Figure 12). These terminal gages represent the most downstream
monitoring locations in their respective sub-watersheds, capturing the
integrated hydrologic response of their contributing areas.

\begin{figure}
\centering
\includegraphics[width=4.36636in,height=4.86131in]{gwbase_manuscript_07/media/image12.png}
\caption{. Overview map}
\end{figure}

After data filtering based on data availability and quality criteria,
only 6 of the 12 terminal gages retained sufficient concurrent
streamflow and water table elevation (WTE) data to conduct the paired
analysis. A representative subbasin is shown in Figure 13. The yellow
star represents the terminal gage location where the Bear River gage is
situated. The orange dots indicate additional streamflow gaging stations
in the upstream network, and the brown squares represent groundwater
monitoring wells distributed throughout the contributing watershed.

\begin{quote}
Bear River gage with upstream catchment
\end{quote}

\begin{enumerate}
\def\labelenumi{\arabic{enumi}.}
\item
  Overall ΔWTE--ΔQ Relationships
\end{enumerate}

To quantify the strength and predictability of groundwater-streamflow
coupling during baseflow-dominated periods, we performed linear
regression analysis on all gage-well pairs using the relationship
between change in streamflow (ΔQ) and change in water table elevation
(ΔWTE). For each well, ΔWTE represents the deviation from the initial
measured water table elevation, calculated after applying piecewise
cubic Hermite interpolating polynomial (PCHIP) interpolation to ensure
uniform daily time steps and ΔQ represents the change in baseflow from
the baseline. The regression model takes the form:

ΔQ = β₀ + β₁ · ΔWTE + ε

where β₀ is the intercept, β₁ is the slope coefficient representing the
sensitivity of streamflow change to groundwater level change, and ε is
the residual error. The coefficient of determination (R²) quantifies the
proportion of variance in ΔQ explained by ΔWTE:

R² = 1 - (SS\_res / SS\_tot)

SS\_res = Σᵢ (ΔQᵢ - ΔQ̂ᵢ)²

SS\_tot = Σᵢ (ΔQᵢ - ΔQ̄)²

Figure X presents scatter plots of ΔQ versus ΔWTE for all
baseflow-dominated observations across the six analyzable gages in the
GSLB. Each point represents a daily paired observation of concurrent
streamflow change and groundwater level change from a single well-gage
combination during periods classified as baseflow-dominated by our
machine learning model. The overall analysis pooled 429,009 observations
from 106 unique wells across 6 gages with sufficient data (Table X).

\begin{longtable}[]{@{}
  >{\raggedright\arraybackslash}p{(\linewidth - 6\tabcolsep) * \real{0.1703}}
  >{\raggedright\arraybackslash}p{(\linewidth - 6\tabcolsep) * \real{0.5088}}
  >{\raggedright\arraybackslash}p{(\linewidth - 6\tabcolsep) * \real{0.1710}}
  >{\raggedright\arraybackslash}p{(\linewidth - 6\tabcolsep) * \real{0.1499}}@{}}
\toprule\noalign{}
\begin{minipage}[b]{\linewidth}\raggedright
Gage id
\end{minipage} & \begin{minipage}[b]{\linewidth}\raggedright
gage name
\end{minipage} & \begin{minipage}[b]{\linewidth}\raggedright
slope
\end{minipage} & \begin{minipage}[b]{\linewidth}\raggedright
r\_squared
\end{minipage} \\
\midrule\noalign{}
\endhead
\bottomrule\noalign{}
\endlastfoot
10126000 & BEAR RIVER NEAR CORINNE - UT & 1.071 & 0.001 \\
10141000 & WEBER RIVER NEAR PLAIN CITY - UT & 0.304 & 0.014 \\
10143500 & CENTERVILLE CREEK ABV. DIV NEAR CENTERVILLE - UT & 0.015 &
0.063 \\
10152000 & SPANISH FORK NEAR LAKE SHORE - UTAH & -0.747 & 0.003 \\
10163000 & PROVO RIVER AT PROVO - UT & 0.659 & 0.004 \\
10168000 & LITTLE COTTONWOOD CREEK @ JORDAN RIVER NR SLC & 0.025 &
0.002 \\
\end{longtable}

The Bear River gage (ID: 10126000) exhibits a moderate positive linear
relationship with R² = 0.001 and slope = 1.071 (p \textless{} 0.001),
indicating statistically significant coupling despite the low R² value.
The low R² but significant p-value suggests that while a linear trend
exists, substantial scatter arises from measurement noise, well
heterogeneity, and unmodeled hydrologic processes (e.g., lateral inflow
variations, transient storage effects). The positive slope confirms the
expected physical relationship: declining groundwater levels (negative
ΔWTE) correspond to decreasing baseflow (negative ΔQ), and vice versa.

\includegraphics[width=4.79389in,height=2.55418in]{gwbase_manuscript_07/media/image14.png}

\begin{enumerate}
\def\labelenumi{\arabic{enumi}.}
\item
  Monthly Variation in ΔWTE--ΔQ Coupling
\end{enumerate}

Monthly stratification of the regression analysis reveals pronounced
temporal variability in coupling strength, reflecting seasonal shifts in
hydrologic regime and aquifer-stream connectivity (Figure Y, Table Y).
For the Bear River gage, monthly R² values range from near-zero in
winter/spring months to peak values exceeding 0.10 in late
summer/autumn.

\begin{longtable}[]{@{}
  >{\raggedright\arraybackslash}p{(\linewidth - 14\tabcolsep) * \real{0.1226}}
  >{\raggedright\arraybackslash}p{(\linewidth - 14\tabcolsep) * \real{0.1279}}
  >{\raggedright\arraybackslash}p{(\linewidth - 14\tabcolsep) * \real{0.1034}}
  >{\raggedright\arraybackslash}p{(\linewidth - 14\tabcolsep) * \real{0.1456}}
  >{\raggedright\arraybackslash}p{(\linewidth - 14\tabcolsep) * \real{0.1251}}
  >{\raggedright\arraybackslash}p{(\linewidth - 14\tabcolsep) * \real{0.1251}}
  >{\raggedright\arraybackslash}p{(\linewidth - 14\tabcolsep) * \real{0.1251}}
  >{\raggedright\arraybackslash}p{(\linewidth - 14\tabcolsep) * \real{0.1251}}@{}}
\caption{Table 1. INSERT CAPTION}\tabularnewline
\toprule\noalign{}
\begin{minipage}[b]{\linewidth}\raggedright
Gage id
\end{minipage} & \begin{minipage}[b]{\linewidth}\raggedright
month
\end{minipage} & \begin{minipage}[b]{\linewidth}\raggedright
wells
\end{minipage} & \begin{minipage}[b]{\linewidth}\raggedright
measurements
\end{minipage} & \begin{minipage}[b]{\linewidth}\raggedright
slope
\end{minipage} & \begin{minipage}[b]{\linewidth}\raggedright
intercept
\end{minipage} & \begin{minipage}[b]{\linewidth}\raggedright
r\_squared
\end{minipage} & \begin{minipage}[b]{\linewidth}\raggedright
p\_value
\end{minipage} \\
\midrule\noalign{}
\endfirsthead
\toprule\noalign{}
\begin{minipage}[b]{\linewidth}\raggedright
Gage id
\end{minipage} & \begin{minipage}[b]{\linewidth}\raggedright
month
\end{minipage} & \begin{minipage}[b]{\linewidth}\raggedright
wells
\end{minipage} & \begin{minipage}[b]{\linewidth}\raggedright
measurements
\end{minipage} & \begin{minipage}[b]{\linewidth}\raggedright
slope
\end{minipage} & \begin{minipage}[b]{\linewidth}\raggedright
intercept
\end{minipage} & \begin{minipage}[b]{\linewidth}\raggedright
r\_squared
\end{minipage} & \begin{minipage}[b]{\linewidth}\raggedright
p\_value
\end{minipage} \\
\midrule\noalign{}
\endhead
\bottomrule\noalign{}
\endlastfoot
10126000 & January & 382 & 32038 & 0.42329408 & 356.719108 & 0.00017676
& 0.01732403 \\
10126000 & February & 363 & 16804 & 0.20330711 & 333.477274 & 5.21E-05 &
0.34956104 \\
10126000 & March & 225 & 2418 & 1.9782467 & 337.595215 & 0.00138637 &
0.06716006 \\
10126000 & April & 304 & 4343 & 0.1895384 & -142.50146 & 4.32E-05 &
0.66494589 \\
10126000 & May & 372 & 34812 & 0.73773674 & -193.31561 & 0.00069874 &
8.11E-07 \\
10126000 & June & 380 & 80064 & 1.11422373 & -182.76151 & 0.00142193 &
1.35E-26 \\
10126000 & July & 436 & 172418 & 1.51729829 & -188.98326 & 0.00249797 &
8.79E-96 \\
10126000 & August & 469 & 174632 & 1.80305046 & -148.83175 & 0.0026969 &
1.44E-104 \\
10126000 & September & 474 & 144785 & 1.80258499 & -86.751679 &
0.00267814 & 1.97E-86 \\
10126000 & October & 391 & 123941 & 1.83750733 & 169.986498 & 0.00222707
& 4.74E-62 \\
10126000 & November & 379 & 68799 & 1.16313193 & 323.716363 & 0.00126492
& 1.04E-20 \\
10126000 & December & 381 & 49364 & 0.73492964 & 340.770515 & 0.00052904
& 3.21E-07 \\
\end{longtable}

Winter months (December--February) exhibit the weakest coupling, with
January R² = 0.00018 and February R² = 0.00005 (though January remains
statistically significant at p = 0.017). The extremely low R² values
during winter reflect minimal variability in both streamflow and
groundwater levels due to frozen ground conditions, reduced recharge,
and stable recession dynamics. The large sample sizes
(\textgreater16,000 observations per month) provide statistical power to
detect even weak trends, but the practical predictive utility is
negligible.

Spring months (March--May) show intermediate coupling as snowmelt
begins. During this transition period, surface runoff contributions
dilute the baseflow signal, introducing noise into the ΔWTE--ΔQ
relationship. Wells located closer to the channel may respond rapidly to
infiltrating snowmelt, while distant wells exhibit lagged responses,
collectively broadening the scatter in the regression.

Late summer and autumn months (July--October) display the strongest
ΔWTE--ΔQ coupling. During this baseflow-dominated window, streamflow is
sustained almost entirely by groundwater discharge, with minimal
interference from surface runoff or precipitation. The enhanced coupling
demonstrates that groundwater level fluctuations become the primary
control on streamflow variability. The slope coefficients during these
months also tend to be higher and more stable, indicating consistent
linear sensitivity across the aquifer network.

The monthly analysis confirms that aggregating data across the entire
year dilutes the signal present during optimal coupling periods.
Focusing future modeling efforts on the late summer/autumn window would
maximize the predictive power of ΔQ--ΔWTE relationships for baseflow
estimation.

\begin{enumerate}
\def\labelenumi{\arabic{enumi}.}
\setcounter{enumi}{1}
\item
  Seasonal ΔWTE--ΔQ Coupling Patterns
\end{enumerate}

Aggregating monthly data into meteorological seasons (Winter: Dec--Feb,
Spring: Mar--May, Summer: Jun--Aug, Fall: Sep--Nov) provides a coarser
temporal lens for evaluating coupling dynamics while maintaining
sufficient sample sizes for robust statistical inference.

\begin{longtable}[]{@{}
  >{\raggedright\arraybackslash}p{(\linewidth - 16\tabcolsep) * \real{0.0983}}
  >{\centering\arraybackslash}p{(\linewidth - 16\tabcolsep) * \real{0.0962}}
  >{\centering\arraybackslash}p{(\linewidth - 16\tabcolsep) * \real{0.0962}}
  >{\centering\arraybackslash}p{(\linewidth - 16\tabcolsep) * \real{0.1454}}
  >{\centering\arraybackslash}p{(\linewidth - 16\tabcolsep) * \real{0.1128}}
  >{\centering\arraybackslash}p{(\linewidth - 16\tabcolsep) * \real{0.1128}}
  >{\centering\arraybackslash}p{(\linewidth - 16\tabcolsep) * \real{0.1128}}
  >{\centering\arraybackslash}p{(\linewidth - 16\tabcolsep) * \real{0.1128}}
  >{\centering\arraybackslash}p{(\linewidth - 16\tabcolsep) * \real{0.1128}}@{}}
\toprule\noalign{}
\begin{minipage}[b]{\linewidth}\raggedright
gage\_id
\end{minipage} & \begin{minipage}[b]{\linewidth}\centering
season
\end{minipage} & \begin{minipage}[b]{\linewidth}\centering
wells
\end{minipage} & \begin{minipage}[b]{\linewidth}\centering
measurements
\end{minipage} & \begin{minipage}[b]{\linewidth}\centering
slope
\end{minipage} & \begin{minipage}[b]{\linewidth}\centering
intercept
\end{minipage} & \begin{minipage}[b]{\linewidth}\centering
r\_squared
\end{minipage} & \begin{minipage}[b]{\linewidth}\centering
p\_value
\end{minipage} & \begin{minipage}[b]{\linewidth}\centering
std\_err
\end{minipage} \\
\midrule\noalign{}
\endhead
\bottomrule\noalign{}
\endlastfoot
10126000 & Fall & 475 & 337525 & 526.260819 & 33290.994 & 0.00133922 &
2.26E-100 & 24.736169 \\
10126000 & Spring & 381 & 41573 & 203.731292 & -57461.75 & 0.00031677 &
0.00028439 & 56.1331772 \\
10126000 & Summer & 470 & 427114 & 570.619309 & -62594.82 & 0.00236644 &
4.58E-222 & 17.9272309 \\
10126000 & Winter & 383 & 98206 & 194.255359 & 125912.609 & 0.00028991 &
9.49E-08 & 36.4009862 \\
\end{longtable}

Fall exhibits the strongest coupling signal with R² = 0.00134 and slope
= 526.26 (p \textless{} 2.26×10⁻¹⁰⁰), supported by the largest dataset
(337,525 observations from 475 wells). The extremely small p-value
reflects overwhelming statistical significance due to the massive sample
size, even though the R² indicates that only 0.13\% of variance is
explained by the linear relationship. However, this should be
interpreted in context: the scatter arises from spatial heterogeneity
(475 distinct wells with different hydraulic properties and distances to
the stream) rather than weak coupling per se. Individual high-performing
wells within this ensemble likely exhibit much stronger relationships
(R² \textgreater{} 0.3--0.5), but are averaged down when pooled.

Spring displays the second-highest R² (0.00032) despite having an order
of magnitude fewer observations (41,573). This suggests that the wells
active during spring may have more homogeneous hydraulic properties or
more direct stream connectivity, reducing scatter. Alternatively, the
restricted temporal window of spring baseflow periods (shorter duration
between snowmelt peaks) may naturally limit the range of ΔWTE--ΔQ
combinations, artificially tightening the regression.

The seasonal aggregation confirms that Fall (late summer/autumn)
provides the optimal data window for establishing predictive ΔWTE--ΔQ
relationships. This aligns with hydrologic expectations: Fall conditions
feature stable baseflow recession, minimal surface runoff interference,
and sustained groundwater discharge.

\begin{enumerate}
\def\labelenumi{\arabic{enumi}.}
\setcounter{enumi}{2}
\item
  Spatial Distribution of Well--Gage Coupling
\end{enumerate}

To examine spatial heterogeneity in groundwater--streamflow coupling,
well-level R² values were mapped for each terminal gage. Figure
14illustrates the spatial distribution of lagged R² values for the Bear
River gage (ID 10126000), with wells colored by correlation strength.

The map reveals pronounced spatial clustering of high-performing wells.
While the majority of wells exhibit weak correlations (mean R² = 0.032),
a small subset shows substantially stronger coupling, with maximum R²
reaching 0.72. These high-R² wells are predominantly located along major
valley corridors and proximal to the stream network, whereas distal or
upland wells tend to exhibit weak or negligible dependence.

This spatial structure indicates that the low pooled R² values primarily
reflect averaging across hydraulically heterogeneous wells rather than
an absence of groundwater--streamflow coupling.

\includegraphics[width=4.41284in,height=3.38257in]{gwbase_manuscript_07/media/image15.png}

Figure 14 Spatial distribution of well--gage coupling strength for the
Bear River watershed.

Wells are colored by lagged coefficient of determination (R²), with
warmer colors indicating stronger ΔWTE--ΔQ dependence during
baseflow-dominated periods. The yellow star marks the terminal gage
location. Numbered red circles indicate the top 10 wells ranked by R².
Despite low mean R² across all wells, a small subset exhibits strong,
spatially clustered coupling, highlighting pronounced hydrogeologic
heterogeneity within the basin

\begin{enumerate}
\def\labelenumi{\arabic{enumi}.}
\setcounter{enumi}{3}
\item
  High-Performing Wells and Subset Reanalysis
\end{enumerate}

To evaluate whether a small number of hydraulically well-connected wells
disproportionately drive the observed ΔWTE--ΔQ signal, we conducted a
subset analysis using only the top 10 wells ranked by R² for each gage.

Recomputing the ΔQ--ΔWTE regression using this reduced subset
substantially increases explanatory power relative to the pooled
analysis. For the Bear River gage, the mean R² across all wells is
0.032, whereas the top-10 subset yields R² values exceeding X (Figure
Y). This contrast demonstrates that strong groundwater--streamflow
coupling exists locally but is obscured when averaged across
heterogeneous well populations.

These results suggest that identifying and weighting high-performing
wells may significantly improve baseflow estimation and
groundwater--streamflow attribution.

\begin{enumerate}
\def\labelenumi{\arabic{enumi}.}
\setcounter{enumi}{1}
\item
  Mutual information+
\end{enumerate}

Figure X presents the spatial distribution of mutual information (MI)
values for all well-gage pairs in the Bear River catchment during
baseflow-dominated periods. The map reveals pronounced spatial
heterogeneity in information coupling strength, with MI values ranging
from near-zero to approximately 0.5 bits. Wells are color-coded
according to their MI values, with warmer colors indicating higher
information sharing with streamflow variations. \hl{(?}

\includegraphics[width=6.25826in,height=4.69904in]{gwbase_manuscript_07/media/image16.png}

Conversely, wells situated in coarse alluvial deposits immediately
adjacent to the channel exhibit both high MI and proportionally high R²
values. These wells show MI values of 0.30 to 0.48 bits coupled with R²
values exceeding 0.40, closely following the theoretical relationship
expected for bivariate Gaussian distributions. The concordance between
MI and R² confirms that aquifer-stream coupling in these settings is
approximately linear and can be effectively modeled using simple
regression approaches. The high information content combined with linear
functional form makes these wells optimal candidates for operational
baseflow estimation.

\begin{enumerate}
\def\labelenumi{\arabic{enumi}.}
\setcounter{enumi}{2}
\item
  Lagged versus Unlagged Coupling Metrics
\end{enumerate}

The choice between using concurrent (zero-lag) versus time-shifted
(optimal-lag) groundwater observations for baseflow estimation involves
trade-offs between model simplicity, data requirements, and predictive
performance. While zero-lag approaches offer operational simplicity and
immediate applicability to real-time monitoring, optimal-lag approaches
may capture additional coupling strength in wells where hydraulic
signals propagate with measurable time delays. To quantify these
trade-offs systematically, we conducted a comprehensive comparison of
coupling metrics computed under zero-lag versus optimal-lag
configurations across all well-gage pairs in the Bear River basin.

\begin{longtable}[]{@{}
  >{\raggedright\arraybackslash}p{(\linewidth - 12\tabcolsep) * \real{0.1417}}
  >{\raggedright\arraybackslash}p{(\linewidth - 12\tabcolsep) * \real{0.0986}}
  >{\raggedright\arraybackslash}p{(\linewidth - 12\tabcolsep) * \real{0.1916}}
  >{\raggedright\arraybackslash}p{(\linewidth - 12\tabcolsep) * \real{0.0747}}
  >{\raggedright\arraybackslash}p{(\linewidth - 12\tabcolsep) * \real{0.1713}}
  >{\raggedright\arraybackslash}p{(\linewidth - 12\tabcolsep) * \real{0.1926}}
  >{\raggedright\arraybackslash}p{(\linewidth - 12\tabcolsep) * \real{0.1295}}@{}}
\toprule\noalign{}
\begin{minipage}[b]{\linewidth}\raggedright
lag\_period
\end{minipage} & \begin{minipage}[b]{\linewidth}\raggedright
total\_observations
\end{minipage} & \begin{minipage}[b]{\linewidth}\raggedright
total\_unique\_wells
\end{minipage} & \begin{minipage}[b]{\linewidth}\raggedright
gages
\end{minipage} & \begin{minipage}[b]{\linewidth}\raggedright
avg\_observations\_per\_gage
\end{minipage} & \begin{minipage}[b]{\linewidth}\raggedright
avg\_wells\_per\_gage
\end{minipage} & \begin{minipage}[b]{\linewidth}\raggedright
data\_retention\_pct
\end{minipage} \\
\midrule\noalign{}
\endhead
\bottomrule\noalign{}
\endlastfoot
no\_lag & 429009 & 106 & 4 & 107252.25 & 28 & \\
1\_year & 1493879 & 879 & 6 & 248979.833 & 146.5 & 348.216238 \\
2\_year & 1362752 & 762 & 6 & 227125.333 & 127 & 317.651145 \\
3\_year & 1171455 & 721 & 6 & 195242.5 & 120.166667 & 273.060705 \\
6\_month\_lag & 977131 & 891 & 6 & 162855.167 & 148.5 & 227.764686 \\
3\_month\_lag & 1174309 & 932 & 6 & 195718.167 & 155.333333 &
273.725959 \\
\end{longtable}

For each well-gage pair, we computed three coupling metrics under both
configurations: linear correlation coefficient (R), coefficient of
determination (R²), and mutual information (MI). The zero-lag
configuration uses concurrent daily observations of ΔQ(t) and ΔWTE(t).
The optimal-lag configuration time-shifts the ΔWTE series by k* days,
where k* is the lag that maximizes the correlation coefficient, yielding
comparisons between ΔQ(t) and ΔWTE(t-k*). By examining the distribution
of metric improvements and their spatial and hydrogeologic correlates,
we aim to identify which wells benefit substantially from lag correction
and which can be adequately characterized using simpler zero-lag
formulations.

\begin{enumerate}
\def\labelenumi{\arabic{enumi}.}
\setcounter{enumi}{3}
\item
  Cross-correlation function (CCF) analysis
\end{enumerate}

We computed CCF over a lag window of ±60 days for all well-gage pairs
with sufficient baseflow-dominated observations. For each pair, we
identified the lag value that maximizes the absolute correlation
coefficient, and recorded both the optimal lag and the corresponding
maximum correlation strength. This dual characterization enables
assessment of both the timescale of aquifer-stream interaction and the
strength of time-shifted coupling.

Figure X presents the distribution of optimal lag values across all
analyzed well-gage pairs in the Bear River basin. The histogram reveals
a strongly peaked distribution centered near zero lag, with substantial
asymmetry toward positive lags.

\begin{quote}
\includegraphics[width=6.5in,height=1.92014in]{gwbase_manuscript_07/media/image17.png}

ADD A CAPTION
\end{quote}

Approximately 68\% of well-gage pairs achieve maximum correlation at
zero lag, indicating that the dominant mode of coupling operates on
timescales faster than the daily measurement resolution. This rapid
response is characteristic of well-connected alluvial aquifers where
hydraulic diffusivity is sufficiently high that pressure signals
propagate nearly instantaneously over distances of several hundred
meters to a few kilometers. For these wells, same-day groundwater
observations contain the maximum information about concurrent streamflow
variations, and incorporating time lags provides no improvement in
predictive power.

The remaining 32\% of well-gage pairs display optimal lags ranging from
1 to 45 days, with a median optimal lag of approximately 14 days among
this lagged subset. The distribution of non-zero lags is positively
skewed, with 90\% of lagged wells showing optimal lags between 5 and 30
days. This timescale range is consistent with diffusive propagation of
hydraulic head changes through lower-permeability valley-fill deposits
or across longer travel distances. Wells exhibiting optimal lags
exceeding 30 days are relatively rare, comprising less than 5\% of the
total population, and likely represent either very distant wells or
wells screened in low-transmissivity units that respond sluggishly to
stream stage variations.

Negative lags, where streamflow appears to lead groundwater levels, are
observed in approximately 8\% of well-gage pairs with optimal lags
between -1 and -7 days. While such relationships seem physically
counterintuitive for baseflow conditions, they can arise from several
mechanisms. In gaining stream reaches where groundwater sustains
baseflow, upstream precipitation events may simultaneously increase both
streamflow and aquifer recharge, with the streamflow response appearing
slightly faster due to rapid shallow subsurface flow paths.
Alternatively, negative lags may reflect transient bank storage effects
where rising stream stage induces temporary aquifer storage that
subsequently drains back to the channel. These negative-lag wells
represent a small fraction of the population and require careful
interpretation of their hydrogeologic context.

\begin{enumerate}
\def\labelenumi{\arabic{enumi}.}
\setcounter{enumi}{4}
\item
  Basin-Scale ΔWTE--ΔQ Relationships
\end{enumerate}

To evaluate groundwater--streamflow coupling at the basin scale, ΔQ and
ΔWTE were aggregated across all analyzable terminal gages within the
Great Salt Lake Basin (GSLB). For each gage, ΔQ and ΔWTE were first
computed relative to the gage-specific baseline and then summed to
obtain basin-wide daily changes during baseflow-dominated periods.

Table 2summarizes the regression results for individual gages
contributing to the basin-scale signal. While gage-level slopes vary in
magnitude and sign, all relationships are statistically significant due
to the large number of observations. The basin-scale aggregation
integrates these heterogeneous local responses into a single
system-level signal.

Table 2 Individual terminal gages exhibit weak explanatory power when
evaluated.

\begin{longtable}[]{@{}
  >{\raggedright\arraybackslash}p{(\linewidth - 8\tabcolsep) * \real{0.1386}}
  >{\raggedright\arraybackslash}p{(\linewidth - 8\tabcolsep) * \real{0.4364}}
  >{\raggedright\arraybackslash}p{(\linewidth - 8\tabcolsep) * \real{0.1670}}
  >{\raggedright\arraybackslash}p{(\linewidth - 8\tabcolsep) * \real{0.1167}}
  >{\raggedright\arraybackslash}p{(\linewidth - 8\tabcolsep) * \real{0.1413}}@{}}
\toprule\noalign{}
\begin{minipage}[b]{\linewidth}\raggedright
gage\_id\hspace{0pt}
\end{minipage} & \begin{minipage}[b]{\linewidth}\raggedright
Gage~name\hspace{0pt}
\end{minipage} & \begin{minipage}[b]{\linewidth}\raggedright
slope\hspace{0pt}
\end{minipage} & \begin{minipage}[b]{\linewidth}\raggedright
r\_squared\hspace{0pt}
\end{minipage} & \begin{minipage}[b]{\linewidth}\raggedright
p\_value\hspace{0pt}
\end{minipage} \\
\midrule\noalign{}
\endhead
\bottomrule\noalign{}
\endlastfoot
10126000\hspace{0pt} & BEAR RIVER NEAR CORINNE - UT\hspace{0pt} &
-0.1719403\hspace{0pt} & 0.00\hspace{0pt} & 0.00\hspace{0pt} \\
10141000\hspace{0pt} & WEBER RIVER NEAR PLAIN CITY - UT\hspace{0pt} &
0.30830181\hspace{0pt} & 0.00\hspace{0pt} & 0.00\hspace{0pt} \\
10152000\hspace{0pt} & SPANISH FORK NEAR LAKE SHORE - UTAH\hspace{0pt} &
0.0730415\hspace{0pt} & 0.00\hspace{0pt} & 0.00\hspace{0pt} \\
10163000\hspace{0pt} & PROVO RIVER AT PROVO - UT\hspace{0pt} &
-0.1039531\hspace{0pt} & 0.00\hspace{0pt} & 0.00\hspace{0pt} \\
\end{longtable}

Figure 15 shows the basin-scale relationship between normalized ΔQ and
ΔWTE pooled across all contributing gages. The aggregated dataset
reveals a weak but statistically significant linear trend, indicating a
coherent basin-scale response despite substantial scatter at daily
resolution.

~\includegraphics[width=4.86in,height=2.9918in]{gwbase_manuscript_07/media/image18.jpeg}

Figure 15 Basin-scale relationship between normalized ΔQ and ΔWTE

\begin{enumerate}
\def\labelenumi{\arabic{enumi}.}
\setcounter{enumi}{4}
\item
  Discussion

  \begin{enumerate}
  \def\labelenumii{\arabic{enumii}.}
  \item
    Interpreting Weak Local ΔWTE--ΔQ Relationships
  \end{enumerate}
\end{enumerate}

Across individual gages and pooled well--gage pairs, the ΔWTE--ΔQ
relationships exhibit consistently low coefficients of determination. At
first glance, these low R² values may appear to suggest weak
groundwater--streamflow coupling. However, such an interpretation would
be misleading for large, heterogeneous river basins such as the Great
Salt Lake Basin (GSLB).

At daily resolution, streamflow during baseflow-dominated periods is
influenced not only by regional groundwater storage but also by a range
of superimposed processes, including lateral inflows, transient bank
storage, delayed drainage from hillslopes, and anthropogenic regulation.
Groundwater level measurements further reflect local aquifer conditions
that vary substantially in depth, hydraulic conductivity, and degree of
hydraulic connection to the stream. When these heterogeneous signals are
combined across hundreds of wells, substantial scatter is expected even
in the presence of a physically meaningful groundwater contribution to
baseflow.

Importantly, the statistical significance observed across nearly all
regressions reflects the large sample sizes rather than strong
predictive skill. This distinction underscores that low R² values in
this context should not be interpreted as an absence of coupling, but
rather as a manifestation of spatial and temporal heterogeneity that
obscures simple linear relationships at fine temporal scales.

\begin{enumerate}
\def\labelenumi{\arabic{enumi}.}
\setcounter{enumi}{1}
\item
  Spatial Heterogeneity and Localized Groundwater--Stream Connectivity
\end{enumerate}

Spatial analyses provide critical insight into the structure underlying
the weak pooled relationships. Mapping well-level dependence metrics
reveals pronounced spatial heterogeneity, with strong clustering of
high-performing wells near major stream corridors and within valley-fill
alluvial aquifers. These wells exhibit substantially higher R² and
mutual information values compared to the basin-wide average, indicating
direct and responsive hydraulic connection to the stream network.

Conversely, wells located farther from channels, screened in
lower-permeability units, or situated in upland settings tend to show
weak or negligible dependence. This pattern is consistent with
conceptual models of gaining streams in semi-arid basins, where only a
subset of the aquifer system actively contributes to baseflow over short
timescales.

The existence of spatially coherent clusters of high-performing wells
demonstrates that groundwater--streamflow coupling is not uniformly
distributed across the basin. Instead, it is highly localized and
controlled by hydrogeologic setting. Aggregating across all wells
therefore dilutes strong local signals, emphasizing the importance of
spatial context when interpreting basin-scale statistics.

\begin{enumerate}
\def\labelenumi{\arabic{enumi}.}
\setcounter{enumi}{2}
\item
  Linear Versus Nonlinear Dependence and the Role of Mutual Information
\end{enumerate}

While linear regression captures first-order sensitivity between ΔWTE
and ΔQ, it inherently assumes a linear functional form that may not hold
across all hydrogeologic settings. Mutual information analysis provides
a complementary perspective by detecting dependence structures that are
nonlinear, threshold-based, or intermittent.

The comparison between R² and MI highlights several wells with moderate
to high MI but weak linear correlation. These wells likely reflect
nonlinear storage--discharge relationships, episodic hydraulic
connection, or threshold behavior during periods of declining
groundwater levels. In such cases, streamflow response may only become
sensitive to groundwater decline once water tables fall below critical
elevations, producing dependence that is poorly captured by linear
regression.

The combined use of R² and MI therefore provides a more complete
characterization of groundwater--streamflow interactions. Linear metrics
identify wells suitable for simple predictive modeling, while MI
highlights wells that remain hydrologically informative despite
nonlinear behavior. Together, these measures reinforce that groundwater
influence on baseflow cannot be fully described by a single statistical
metric.

\begin{enumerate}
\def\labelenumi{\arabic{enumi}.}
\setcounter{enumi}{3}
\item
  Temporal Structure and the Significance of Lagged Responses
\end{enumerate}

Lag analysis further clarifies the temporal dynamics of
groundwater--streamflow coupling. For the majority of well--gage pairs,
maximum correlation occurs at zero lag, indicating that pressure signals
propagate rapidly between the aquifer and stream at daily resolution.
Such behavior is characteristic of well-connected alluvial systems with
high hydraulic diffusivity.

A substantial minority of wells, however, exhibit optimal lags ranging
from several days to several weeks. These delayed responses are
consistent with diffusive propagation of hydraulic head changes through
lower-permeability materials or across greater distances from the
channel. The presence of lagged coupling highlights that groundwater
contributions to baseflow operate across a spectrum of timescales, even
within a single watershed.

Importantly, incorporating lag improves dependence metrics for only a
subset of wells, suggesting diminishing returns relative to increased
data requirements and model complexity. This finding supports the
practical utility of zero-lag formulations for large-scale screening
analyses, while acknowledging that lagged approaches may be valuable for
site-specific investigations.

\begin{enumerate}
\def\labelenumi{\arabic{enumi}.}
\setcounter{enumi}{4}
\item
  Emergence of Basin-Scale Signal Through Aggregation
\end{enumerate}

Although individual gage-level relationships remain weak, aggregation
across terminal gages reveals a coherent basin-scale ΔWTE--ΔQ signal.
Summing ΔQ and ΔWTE across watersheds suppresses local noise and
emphasizes shared regional trends driven by long-term groundwater
storage changes.

This behavior is analogous to signal emergence in climate and hydrologic
trend analysis, where aggregation across space or time reveals patterns
that are not detectable at individual sites. In the GSLB, basin-scale
aggregation integrates heterogeneous local responses into a system-level
signal consistent with regional groundwater decline and reduced baseflow
contributions.

The emergence of a basin-scale response despite weak local
predictability highlights the value of multi-scale analysis. While daily
streamflow at individual gages may not be reliably predicted from
groundwater levels alone, collective behavior across the basin provides
diagnostic evidence of groundwater influence on surface water
availability.

\begin{enumerate}
\def\labelenumi{\arabic{enumi}.}
\setcounter{enumi}{5}
\item
  Conclusions
\end{enumerate}

This section is not mandatory but can be added to the manuscript if the
discussion is unusually long or complex.

\textbf{Supplementary Materials:} The following supporting information
can be downloaded at: https://www.mdpi.com/article/doi/s1, Figure S1:
title; Table S1: title; Video S1: title.

\textbf{Author Contributions:} For research articles with several
authors, a short paragraph specifying their individual contributions
must be provided. The following statements should be used
``Conceptualization, X.X. and Y.Y.; methodology, X.X.; software, X.X.;
validation, X.X., Y.Y. and Z.Z.; formal analysis, X.X.; investigation,
X.X.; resources, X.X.; data curation, X.X.; writing---original draft
preparation, X.X.; writing---review and editing, X.X.; visualization,
X.X.; supervision, X.X.; project administration, X.X.; funding
acquisition, Y.Y. All authors have read and agreed to the published
version of the manuscript.'' Please turn to the
\href{https://img.mdpi.org/data/contributor-role-instruction.pdf}{CRediT
taxonomy} for the term explanation. Authorship must be limited to those
who have contributed substantially to the work reported.

\textbf{Funding:} Please add: ``This research received no external
funding'' or ``This research was funded by NAME OF FUNDER, grant number
XXX'' and ``The APC was funded by XXX''. Check carefully that the
details given are accurate and use the standard spelling of funding
agency names at https://search.crossref.org/funding. Any errors may
affect your future funding.

\textbf{Data Availability Statement:} We encourage all authors of
articles published in MDPI journals to share their research data. In
this section, please provide details regarding where data supporting
reported results can be found, including links to publicly archived
datasets analyzed or generated during the study. Where no new data were
created, or where data is unavailable due to privacy or ethical
restrictions, a statement is still required. Suggested Data Availability
Statements are available in section ``MDPI Research Data Policies'' at
https://www.mdpi.com/ethics.

\textbf{Acknowledgments:} In this section, you can acknowledge any
support given which is not covered by the author contribution or funding
sections. This may include administrative and technical support, or
donations in kind (e.g., materials used for experiments). Where GenAI
has been used for purposes such as generating text, data, or graphics,
or for study design, data collection, analysis, or interpretation of
data, please add ``During the preparation of this manuscript/study, the
author(s) used {[}tool name, version information{]} for the purposes of
{[}description of use{]}. The authors have reviewed and edited the
output and take full responsibility for the content of this
publication.''

\textbf{Conflicts of Interest:} Declare conflicts of interest or state
``The authors declare no conflicts of interest.'' Authors must identify
and declare any personal circumstances or interest that may be perceived
as inappropriately influencing the representation or interpretation of
reported research results. Any role of the funders in the design of the
study; in the collection, analyses or interpretation of data; in the
writing of the manuscript; or in the decision to publish the results
must be declared in this section. If there is no role, please state
``The funders had no role in the design of the study; in the collection,
analyses, or interpretation of data; in the writing of the manuscript;
or in the decision to publish the results''.

\begin{enumerate}
\def\labelenumi{\arabic{enumi}.}
\setcounter{enumi}{6}
\item
  Abbreviations
\end{enumerate}

The following abbreviations are used in this manuscript:

\begin{longtable}[]{@{}
  >{\raggedright\arraybackslash}p{(\linewidth - 2\tabcolsep) * \real{0.1191}}
  >{\raggedright\arraybackslash}p{(\linewidth - 2\tabcolsep) * \real{0.8809}}@{}}
\toprule\noalign{}
\begin{minipage}[b]{\linewidth}\raggedright
MDPI
\end{minipage} & \begin{minipage}[b]{\linewidth}\raggedright
Multidisciplinary Digital Publishing Institute
\end{minipage} \\
\midrule\noalign{}
\endhead
\bottomrule\noalign{}
\endlastfoot
DOAJ & Directory of open access journals \\
TLA & Three letter acronym \\
LD & Linear dichroism \\
\end{longtable}

\begin{enumerate}
\def\labelenumi{\arabic{enumi}.}
\setcounter{enumi}{7}
\item
  References
\end{enumerate}

1. Carroll, R.W.H.; Niswonger, R.G.; Ulrich, C.; Varadharajan, C.;
Siirila-Woodburn, E.R.; Williams, K.H. Declining Groundwater Storage
Expected to Amplify Mountain Streamflow Reductions in a Warmer World.
\emph{Nat Water} \textbf{2024}, \emph{2}, 419--433,
doi:10.1038/s44221-024-00239-0.

2. Mukherjee, A.; Bhanja, S.N.; Wada, Y. Groundwater Depletion Causing
Reduction of Baseflow Triggering Ganges River Summer Drying.
\emph{Scientific Reports} \textbf{2018}, \emph{8}, 12049,
doi:10.1038/s41598-018-30246-7.

3. Bosch, D.D.; Lowrance, R.R.; Sheridan, J.M.; Williams, R.G. Ground
Water Storage Effect on Streamflow for a Southeastern Coastal Plain
Watershed. \emph{Groundwater} \textbf{2003}, \emph{41}, 903--912,
doi:10.1111/j.1745-6584.2003.tb02433.x.

4. Abbas, S.A.; Bailey, R.T.; White, J.T.; Arnold, J.G.; White, M.J.
Estimation of Groundwater Storage Loss Using Surface--Subsurface
Hydrologic Modeling in an Irrigated Agricultural Region.
\emph{Scientific Reports} \textbf{2025}, \emph{15}, 8350,
doi:10.1038/s41598-025-92987-6.

5. Konikow, L.F. Long-Term Groundwater Depletion in the United States.
\emph{Groundwater} \textbf{2015}, \emph{53}, 2--9,
doi:10.1111/gwat.12306.

6. Hodgkins, G.A.; Dudley, R.W.; Nielsen, M.G.; Renard, B.; Qi, S.L.
Groundwater-Level Trends in the U.S. Glacial Aquifer System, 1964--2013.
\emph{Journal of Hydrology} \textbf{2017}, \emph{553}, 289--303,
doi:10.1016/j.jhydrol.2017.07.055.

7. Rojanasakul, M.; Flavelle, C.; Migliozzi, B.; Murray, E. America Is
Draining Its Groundwater Like There's No Tomorrow. \emph{New York Times}
2023.

8. Rojanasakul, M.; Flavelle, C.; Migliozzi, B.; Murray, E. America Is
Using Up Its Groundwater Like There's No Tomorrow. \emph{The New York
Times} 2023.

9. Bedford, D.; Douglass, A. Changing Properties of Snowpack in the
Great Salt Lake Basin, Western United States, from a 26-Year SNO℡
Record. \emph{The Professional Geographer} \textbf{2008}, \emph{60},
374--386, doi:10.1080/00330120802013646.

10. Yeager, K.N.; Steenburgh, W.J.; Alcott, T.I. Contributions of
Lake-Effect Periods to the Cool-Season Hydroclimate of the Great Salt
Lake Basin. \textbf{2013}, doi:10.1175/JAMC-D-12-077.1.

11. USGS USGS Groundwater Data for the Nation Available online:
https://nwis.waterdata.usgs.gov/nwis/gw (accessed on 10 December 2025).

12. USGS Water Data for the Nation Available online:
https://waterdata.usgs.gov/nwis (accessed on 28 December 2025).

13. GEOGLOWS Available online: https://www.geoglows.org/ (accessed on 28
December 2025).

14. Aghababaei, A.; Jones, N.L.; Williams, G.P.; Webster-Esho, E.;
Heijden, R. van der; Li, X.; Clement, T.P.; Rizzo, D.M.; Aghababaei, A.;
Jones, N.L.; et al. Development and Comparison of Methods for
Identification of Baseflow-Dominant Periods in Streamflow Records.
\emph{Water} \textbf{2025}, \emph{17}, doi:10.3390/w17213083.

\begin{enumerate}
\def\labelenumi{\arabic{enumi}.}
\item
  Author 1, A.B.; Author 2, C.D. Title of the article. \emph{Abbreviated
  Journal Name} \textbf{Year}, \emph{Volume}, page range.
\item
  Author 1, A.; Author 2, B. Title of the chapter. In \emph{Book Title},
  2nd ed.; Editor 1, A., Editor 2, B., Eds.; Publisher: Publisher
  Location, Country, 2007; Volume 3, pp. 154--196.
\item
  Author 1, A.; Author 2, B. \emph{Book Title}, 3rd ed.; Publisher:
  Publisher Location, Country, 2008; pp. 154--196.
\end{enumerate}

\begin{enumerate}
\def\labelenumi{\arabic{enumi}.}
\item
  Author 1, A.B.; Author 2, C. Title of Unpublished Work.
  \emph{Abbreviated Journal Name} year, \emph{phrase indicating stage of
  publication (submitted; accepted; in press)}.
\item
  Author 1, A.B. (University, City, State, Country); Author 2, C.
  (Institute, City, State, Country). Personal communication, 2012.
\item
  Author 1, A.B.; Author 2, C.D.; Author 3, E.F. Title of Presentation.
  In Proceedings of the Name of the Conference, Location of Conference,
  Country, Date of Conference (Day Month Year).
\item
  Author 1, A.B. Title of Thesis. Level of Thesis, Degree-Granting
  University, Location of University, Date of Completion.
\item
  Title of Site. Available online: URL (accessed on Day Month Year).
\end{enumerate}

\textbf{Disclaimer/Publisher's Note:} The statements, opinions and data
contained in all publications are solely those of the individual
author(s) and contributor(s) and not of MDPI and/or the editor(s). MDPI
and/or the editor(s) disclaim responsibility for any injury to people or
property resulting from any ideas, methods, instructions or products
referred to in the content.

\end{document}
